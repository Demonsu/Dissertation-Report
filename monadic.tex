\chapter{Monadic Parser}

\section{Introduction}

In functional programming, a popular approach to build recursive descent parsers is to model parsers as functions, and to define higher-order functions (or combinators) that implement grammar constructions such as sequencing, choice, and repeatition. Wadler \cite{Wadler:1992} has been realised that using monad, an algebraic structure for mathematics, to build parsers would brings lots practical benefics. For example, using a monadic sequencing combinators for parsers avoids the messy manipulation of nested tuples of results present in earlier work. Moreover, using \textit{monad comprehension} notation makes parser more compact and easy to read.

A monadic parser could be expressed in a modular way in terms of two simpler monads, so that the basic parser combinators no longer need to be defined explicitly. Rather, they arise automatically as a special case of lifting monad operations from a base monad \textit{m} to a certain other monad parameterised over \textit{m}.
