\chapter{Discussion}

F2J is a new born programming language. While we using it to build the parser and pretty printer library, we have encountered a lot of problems. Most of them were solved with the members in HKU programming group, but still could be improved:

\begin{itemize}
\item Module System

Currently F2J only supports single file compilation, which means that I have to put all the codes in one file. It is hard to manage the project and reuse most of the codes. Also, having module system could

\item Exceptions

F2J does not supports exceptions, so that I cannot use those Java functions that will throw exceptions, such as I/O libraries. Currently we could make use of the power of Java interpolation of F2J by writing a bridge class in Java to call those fuctions that may throw exceptions, but that just a temporary solution.

\item Type Inferences

F2J's type parameters must be applied explicitly, which makes the code looks too verbose and hard to read.

\item Macros

F2J does not support macros now, but it is useful sometimes. For example, when I was using \texttt{Thunk} in the library, I have to construct a \texttt{Thunk} explicitly, such as

\begin{lstlisting}
let singleton[A] (x : A) : PList[A] =
    Cons[A] x (\(__: Unit) -> (Nil[A]));
\end{lstlisting}

If F2J supports macros, the lambda function could be replaced by a more expressive macro.

\end{itemize}
